\documentclass{article}

\usepackage[english]{babel}
\usepackage[utf8]{inputenc}
\usepackage{amsmath,amssymb}
\usepackage{parskip}
\usepackage{graphicx}

% Margins
\usepackage[top=2.5cm, left=3cm, right=3cm, bottom=4.0cm]{geometry}
% Colour table cells
\usepackage[table]{xcolor}

% Get larger line spacing in table
\newcommand{\tablespace}{\\[1.25mm]}
\newcommand\Tstrut{\rule{0pt}{2.6ex}}         % = `top' strut
\newcommand\tstrut{\rule{0pt}{2.0ex}}         % = `top' strut
\newcommand\Bstrut{\rule[-0.9ex]{0pt}{0pt}}   % = `bottom' strut

%%%%%%%%%%%%%%%%%
%     Title     %
%%%%%%%%%%%%%%%%%
\title{EE2101 \\ Assignment-1}
\author{Kota Pranav Kumar \\ EE19BTECH11051}
\date{\today}

\begin{document}
\maketitle

%%%%%%%%%%%%%%%%%
%   Problem 1   %
%%%%%%%%%%%%%%%%%
\section{Problem }
For the translational mechanical system with a nonlinear spring shown in the following figure,find the transfer function, $G(s) = X(s)/F(s)$ ,for small excursions around $f(t) = 1$.The spring is defined by $ x_s(t) =1 - e^-^\(f_s(t)\) $, where   $x_s(t)$ is the spring displacement and $f_s(t)$ is the spring force.

\includegraphics[width=10cm]{a.png}
\section{Solution}
Given: Small excursions around f(t) = 1\\ \\
Let $ x(t) = x_o + \delta x$ and  $ f(t) = 1 + \delta f$ , where $x_0$ is the point where $f(t) =1$\\

We know that   \(x(t) = x_s(t) =1 - e^-^\(f_s(t)\)\)\\
\implies $f_s(t)$ = -ln(1 - $x(t)$) \\ \\
As excursions are small,let us linearise the above force expression.\\
ln(1 - $x(t)$) - ln(1 - $x_o$) = $-\delta x/1-x_o$\\
i.e. ln(1 - $x(t)$) = ln(1 - $x_o$) - $\delta x/1-x_o$\\ \\ 
At $f(t)$ = 1, $x = x_o$\\
\implies $x_o$ =$ 1 - e^-^1 = 1 - 0.3678 = 0.6321$\\ 
\implies ln(1 - $x(t)$) = 1 - \delta x/0.3678 = 1 + 2.7188\delta x

\newline\\
Force due to piston , $f_p(t) = \frac{dx}{dt}$ \\
\implies $f_p(t)$ = $\frac{d(x_0 +\delta x)}{dt} $ = $\frac{d\delta x}{dt}$\\ 
Applying Newtons Second Law to the system, we get\\
\[f(t) - f_s(t) -f_p(t) = m\frac{d^2x}{dt^2} =m\frac{d^2(x_o + \delta x)}{dt^2} = m\frac{d^2\delta x}{dt^2}\]
\[\implies 1 + \delta f  -1 - 2.7188\delta x - \frac{d\delta x}{dt} = \frac{d^2\delta x}{dt^2} \]
\[\implies \frac{d^2\delta x}{dt^2} + \frac{d\delta x}{dt} +2.7188\delta x = \delta f   \]
$\mathcal{L}\delta x  = X(s)$\\ \\
$\mathcal{L}{\frac{d\delta x}{dt}} = sX(s) -\delta x(0) = sX(s)$\\ \\ 
$\mathcal{L}{{\frac{d^2\delta x}{dt^2} }}  = s^2X(s) - s\delta x(0) -\delta x'(0) =s^2X (s)$\\ \\
$\mathcal{L}\delta f = F(s)$\\\\



Applying Laplace transform to both the sides,we get
\[s^2X(s) +sX(s) +2.7188X(s) = F(s)\]
\[\implies X(s)/F(s) = 1/(s^2 +s + 2.7188) \]

\[\therefore G(s) =X(s)/F(s) = 1/(s^2 +s +2.7188)\]
















\end{document}
